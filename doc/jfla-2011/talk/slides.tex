\documentclass{beamer}
%\documentclass[handout]{beamer}

\usepackage[latin1]{inputenc}
%\usepackage[french]{babel}

\beamertemplatenavigationsymbolsempty

\definecolor{kwblue}{rgb}{0.67,0.12,0.92}
\definecolor{ceruleanblue}{rgb}{0, 0.48, 0.65}
\definecolor{lightpink}{rgb}{1., 0.71, 0.75}
\definecolor{lightblue}{rgb}{0.8,0.8,1}
\definecolor{lightred}{rgb}{1,0.8,0.8}

\let\emph\alert

\begin{document}

\title{Functory : Une biblioth�que de calcul distribu� \\ pour
  Objective Caml}
\author[Jean-Christophe]{Jean-Christophe Filli\^atre \& Kalyan Krishnamani}
\date{JFLA, 31 janvier 2011}

\begin{frame}
  \titlepage
  \pgfimage[height=8mm]{cnrs-logo2}\hfill
  \pgfimage[height=6mm]{saclay}\hfill
  \pgfimage[height=10mm]{lrilogo}\hfill
  \pgfimage[height=8mm]{upsudlogo}
\end{frame}

\begin{frame}\frametitle{Motivation}
  \begin{itemize}
  \item Why platform: a set of tools for deductive program verification
  \item generates numerous verification conditions (VCs)
  \item discharged by various automated theorem provers
  \item typically takes hours to complete
  \end{itemize}


  \begin{itemize}
  \item a few multi-core machines at our disposal
  \item how to make the best possible use of them?
  \end{itemize}
\end{frame}

\begin{frame}\frametitle{A Distributed Computing Library}

  requirements
  \begin{itemize}
  \item ideally in Ocaml
  \item fault tolerance
  \item user-friendly API
  \end{itemize}
\end{frame}

\begin{frame}\frametitle{Basic Design}
  inspired by Google's Map/Reduce (OSDI 2004)


  \begin{center}
    \includegraphics{master_workers_1.mps}
  \end{center}
\end{frame}

\begin{frame}\frametitle{Interface}

  \begin{ocaml}
  val compute : 
    worker:('a -> 'b) -> 
    master:('a * 'c -> 'b -> ('a * 'c) list) -> 
    ('a * 'c) list -> 
    unit
  \end{ocaml}



  \begin{itemize}
  \item a task has type \ocaml{'a * 'c}, its result has type \ocaml{'b}
  \item a completed task may generate new tasks
  \item the whole computation returns when there is no more task
  \end{itemize}
\end{frame}

% TODO: insert a small example here?

\begin{frame}\frametitle{Simple Implementations}
  \begin{description}
  \item[sequential]
    \begin{itemize}
    \item used for debugging and comparison
    \end{itemize}


  \item[multi-core] 
    \begin{itemize}
    \item to utilize multiple cores on a single machine
    \end{itemize}
  \end{description}
\end{frame}

\begin{frame}\frametitle{Network Implementation}
  based on
  \begin{itemize}
  \item TCP/IP client/server architecture
  \item Ocaml's marshaling capabilities
  \end{itemize}
  
  \vfill
  marshaling considerations
  \begin{enumerate}
  \item \emph{same binary}: we can marshal closures
  \item \emph{same version of Ocaml}: we can only marshal values
  \item \emph{otherwise}: we only marshal strings
  \end{enumerate}
\end{frame}

\begin{frame}\frametitle{Three Network Implementations}
\emph{same binary}
\begin{ocaml}
val compute : (* same as before *) ...
\end{ocaml}



\emph{same version of Caml}
    \begin{ocaml}
val Worker.compute : ('a -> 'b) -> unit
val Master.compute : 
  ('a * 'c -> 'b -> ('a * 'c) list) ->
  ('a * 'c) list -> unit
\end{ocaml}



\emph{otherwise}
    \begin{ocaml}
val Worker.compute : (string -> string) -> unit
val Master.compute : 
  (string * 'c -> string -> (string * 'c) list) ->
  (string * 'c) list -> unit
\end{ocaml}
\end{frame}

\begin{frame}\frametitle{Implementation Details: Cores}

  simply uses \texttt{Unix.fork} (no control over scheduling)

  \begin{ocaml}
  open Cores
  let () = set_number_of_cores 4
  \end{ocaml}

  \begin{center}
    \includegraphics{master_workers_cores.mps}
  \end{center}

  master maintains a queue of pending tasks
\end{frame}

\begin{frame}\frametitle{Implementation Details: Netwok}

  \begin{ocaml}
  open Network
  let () = declare_workers ~n:3 "moloch"
  let () = declare_workers ~n:2 "orcus"
  \end{ocaml}

  \begin{center}
    \includegraphics{master_workers_network.mps}
  \end{center}
  each worker behaves as a server, the master being the client
\end{frame}

\begin{frame}\frametitle{Protocol}
  \only<1,5>{master sends a task to a worker}
  \only<2>{worker computes and sends back a result}
  \only<3-4>{master and workers exchange \emph{ping}/\emph{pong} messages}
  \only<6>{in case of a disconnection...}
  \only<7>{the task is \emph{rescheduled} to another worker}
  \only<8>{whenever one completes...}
  \only<9>{the other one is stopped}
  \only<10>{the master is notified when a computation fails}
  \only<11>{at the very end, the master may ask the workers to stop}
  \vfill
  \begin{center}
    \only<1,5>{\includegraphics{master_workers_assign.mps}}
    \only<2>{\includegraphics{master_workers_completed.mps}}
    \only<3>{\includegraphics{master_workers_ping.mps}}
    \only<4>{\includegraphics{master_workers_pong.mps}}
    \only<6>{\includegraphics{master_workers_disconnection.mps}}
    \only<7>{\includegraphics{master_workers_assign_2.mps}}
    \only<8>{\includegraphics{master_workers_completed_2.mps}}
    \only<9>{\includegraphics{master_workers_kill.mps}}
    \only<10>{\includegraphics{master_workers_aborted.mps}}
    \only<11>{\includegraphics{master_workers_stop.mps}}
  \end{center}
  \begin{center}
    \only<1>{\ocaml{Assign(42, f, a)}}
    \only<2>{\ocaml{Completed (42, b)}}
    \only<3>{\ocaml{Ping}}
    \only<4>{\ocaml{Pong}}
    \only<5>{\ocaml{Assign(43, f, a)}}
    \only<6>{\ocaml{}} % disconnection
    \only<7>{\ocaml{Assign(43, f, a)}}
    \only<8>{\ocaml{Completed (43, b)}}
    \only<9>{\ocaml{Kill 43}}
    \only<10>{\ocaml{Aborted 44}}
    \only<11>{\ocaml{Stop}}
  \end{center}
\end{frame}

\begin{frame}\frametitle{}
  fault tolerance, protocol

  \begin{center}
    \includegraphics{state.mps}
  \end{center}
\end{frame}

% here?
\begin{frame}\frametitle{}
  derived API

  (mention that details are in the paper)
\end{frame}

\begin{frame}\frametitle{}
  benchmarks

  SMT solvers first
\end{frame}

\begin{frame}\frametitle{}
  n-queens
\end{frame}

\begin{frame}\frametitle{}
  Mandelbrot
\end{frame}

\begin{frame}\frametitle{}
  matrix multiplication
\end{frame}

\begin{frame}\frametitle{}
  conclusion, future work, etc.
\end{frame}


\end{document}

%%% Local Variables: 
%%% mode: latex
%%% TeX-master: t
%%% End: 

