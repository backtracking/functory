% \documentclass[xcolor=dvipsnames]{beamer}
% % \documentclass[handout]{beamer}
% \usecolortheme[named=Maroon]{structure} 
% \usetheme{Boadilla}
% \setbeamertemplate{beaver}
\documentclass[xcolor=dvipsnames]{beamer} 
\usecolortheme[named=MidnightBlue]{structure} 
\usetheme[height=7mm]{Rochester} 
\usepackage{fontenc}
\renewcommand*\familydefault{\sfdefault}
\usepackage{times}
% \usepackage[latin1]{inputenc}
\usepackage{alltt}
\renewcommand{\iff}{\leftrightarrow}
\newcommand{\tr}{\textcolor{red}}
\newcommand{\tb}{\textcolor{MidnightBlue}}
\newcommand{\intt}{\textcolor{blue}{\textbf{int }}}
\newcommand{\while}{\textcolor{blue}{\textbf{while }}}
\newcommand{\ret}{\textcolor{blue}{\textbf{return }}}
\newcommand{\asrt}{\textcolor{blue}{\textbf{//@assert }}}
\newcommand{\ts}[1]{\scriptsize{#1}}
\newcommand{\hl}[1]{\colorbox{yellow}{#1}}
%\usepackage[french]{babel}

\beamertemplatenavigationsymbolsempty


\bigskip
\definecolor{MyGray}{rgb}{0.86,0.87,0.98}
\makeatletter\newenvironment{bbox}{%
   \begin{lrbox}
     {\@tempboxa}\begin{minipage}
       {\columnwidth}}
     {\end{minipage}
   \end{lrbox}%
   \colorbox{MyGray}{\usebox{\@tempboxa}}
}\makeatother
% \definecolor{MyGray}{rgb}{0.96,0.97,0.98}


\definecolor{kwblue}{rgb}{0.67,0.12,0.92}
\definecolor{ceruleanblue}{rgb}{0, 0.48, 0.65}
\definecolor{lightpink}{rgb}{1., 0.71, 0.75}
\definecolor{lightblue}{rgb}{0.8,0.8,1}
\definecolor{lightred}{rgb}{1,0.8,0.8}

\let\emph\alert
\newcommand{\coeurs}{c\oe urs}

\begin{document}

%% -------------------------------------------------------------------- %%

\title{FUNCTORY \\ A Distributed Computing Library \\ for Objective Caml}

\author[Kalyan]{Jean-Christophe Filli\^{a}tre \& K. Kalyanasundaram\\
  \- \\ 
  CNRS / INRIA Saclay -- \^{I}le-de-France}
\date {}


\bigskip
\begin{frame}
  \titlepage
  \begin{center}
    \includegraphics[scale=0.11]{cnrs-logo2.jpg}
    \hspace{4mm}
    \includegraphics[scale=0.09]{lrilogo.jpg}
    \hspace{4mm}
    \includegraphics[scale=0.09]{upsudlogo.jpg}
    \hspace{4mm}
    \includegraphics[scale=0.4]{inria-saclay.jpeg}
    % \pgfimage[height=6mm]{saclay}
  \end{center}
\end{frame}

%% -------------------------------------------------------------------- %%

\begin{frame}\frametitle{Motivation}
  \begin{itemize}
  \item In our team, we do deductive program verification
  \item Generates numerous verification conditions
  \item Discharged by various automated provers
  \item Typically takes \emph{hours} to complete
  \end{itemize}
  \vspace{1cm}
  \begin{itemize}
  \item Some multi-core machines at our disposal
  \item How to make the best possible use of them?
  \end{itemize}
\end{frame}

%% -------------------------------------------------------------------- %%

\begin{frame}\frametitle {A Distributed Computing Library}
  Requirements
  \begin{itemize}
  \item Fault-tolerance
  \item User-friendly API
  \item In our favorite programming language (OCaml)
  \end{itemize}
\end{frame}

%% -------------------------------------------------------------------- %%

\begin{frame}\frametitle {Basic Design}

Inspired by Google's Map/Reduce (OSDI 2004)

\begin{itemize}
\item Workers in parallel
\item Master
\end{itemize}

\begin{center}
  \includegraphics{master_workers_1.mps}  
\end{center}
\end{frame}

%% -------------------------------------------------------------------- %%

\begin{frame}\frametitle{Outline}
  \begin{columns}
    \column{0.8\textwidth}
    \begin{itemize}
    \item API
      \begin{itemize}
      \item general-purpose \texttt{compute} function
      \item high-level: map/fold operations
      \item low-level: micro-step computations
      \end{itemize}


\bigskip
    \item Deployment Scenarios
      \begin{itemize}
      \item Sequential
      \item Cores
      \item Network
      \end{itemize}
    \end{itemize}

    \column{0.2\textwidth}
  \begin{tabular}{|p{0.2em}|p{0.2em}|p{0.2em}|}
    \hline
    && \\\hline
    && \\\hline
    && \\\hline
  \end{tabular}
\end{columns}
\end{frame}

%% -------------------------------------------------------------------- %%

\begin{frame}\frametitle{A General-Purpose \texttt{compute} Function}
  

\medskip\noindent
{\definecolor{tmpcolor}{rgb}{0.80,0.80,1.00}\colorbox{tmpcolor}{\begin{minipage}{\textwidth}\ttfamily\parindent 0pt

\noindent\hspace*{2.00em}{\color{blue}val}\hspace*{1.22ex}compute\hspace*{1.22ex}\symbol{58}\hspace*{1.22ex}~\linebreak
\noindent\hspace*{2.00em}worker\symbol{58}(\ensuremath{\alpha}\hspace*{1.22ex}\ensuremath{\rightarrow}\hspace*{1.22ex}\ensuremath{\beta})\hspace*{1.22ex}\ensuremath{\rightarrow}\hspace*{1.22ex}~\linebreak
\noindent\hspace*{2.00em}master\symbol{58}(\ensuremath{\alpha}\hspace*{1.22ex}\ensuremath{\times}\hspace*{1.22ex}\ensuremath{\gamma}\hspace*{1.22ex}\ensuremath{\rightarrow}\hspace*{1.22ex}\ensuremath{\beta}\hspace*{1.22ex}\ensuremath{\rightarrow}\hspace*{1.22ex}(\ensuremath{\alpha}\hspace*{1.22ex}\ensuremath{\times}\hspace*{1.22ex}\ensuremath{\gamma})\hspace*{1.22ex}list)\hspace*{1.22ex}\ensuremath{\rightarrow}\hspace*{1.22ex}~\linebreak
\noindent\hspace*{2.00em}(\ensuremath{\alpha}\hspace*{1.22ex}\ensuremath{\times}\hspace*{1.22ex}\ensuremath{\gamma})\hspace*{1.22ex}list\hspace*{1.22ex}\ensuremath{\rightarrow}\hspace*{1.22ex}~\linebreak
\noindent\hspace*{2.00em}unit
\end{minipage}}}


\medskip\noindent
  \begin{itemize}
  \item A task is of type \textsf{\ensuremath{\alpha}\hspace*{1.22ex}\ensuremath{\times}\hspace*{1.22ex}\ensuremath{\gamma}}, its result of type
    \textsf{\ensuremath{\beta}}
  \item A completed task may in turn generate new tasks
  \item \texttt{compute} returns when there is no more task
  \end{itemize}
\end{frame}

%% -------------------------------------------------------------------- %%

\begin{frame}\frametitle {High-Level API}
  \begin{itemize}
  \item most common map/fold operations over lists
  \end{itemize}
  

\medskip\noindent
{\definecolor{tmpcolor}{rgb}{0.80,0.80,1.00}\colorbox{tmpcolor}{\begin{minipage}{\textwidth}\ttfamily\parindent 0pt

\noindent\hspace*{0.00em}{\color{blue}val}\hspace*{1.22ex}map\symbol{58}\hspace*{1.22ex}f\symbol{58}(\ensuremath{\alpha}\hspace*{1.22ex}\ensuremath{\rightarrow}\hspace*{1.22ex}\ensuremath{\beta})\hspace*{1.22ex}\ensuremath{\rightarrow}\hspace*{1.22ex}\ensuremath{\alpha}\hspace*{1.22ex}list\hspace*{1.22ex}\ensuremath{\rightarrow}\hspace*{1.22ex}\ensuremath{\beta}\hspace*{1.22ex}list~\linebreak
\noindent\hspace*{0.00em}{\color{blue}val}\hspace*{1.22ex}map\_{}fold\symbol{58}\hspace*{1.22ex}f\symbol{58}(\ensuremath{\alpha}\hspace*{1.22ex}\ensuremath{\rightarrow}\hspace*{1.22ex}\ensuremath{\beta})\hspace*{1.22ex}\ensuremath{\rightarrow}\hspace*{1.22ex}~\linebreak
\noindent\hspace*{7.00em}fold\symbol{58}(\ensuremath{\gamma}\hspace*{1.22ex}\ensuremath{\rightarrow}\hspace*{1.22ex}\ensuremath{\beta}\hspace*{1.22ex}\ensuremath{\rightarrow}\hspace*{1.22ex}\ensuremath{\gamma})\hspace*{1.22ex}\ensuremath{\rightarrow}\hspace*{1.22ex}\ensuremath{\gamma}\hspace*{1.22ex}\ensuremath{\rightarrow}\hspace*{1.22ex}\ensuremath{\alpha}\hspace*{1.22ex}list\hspace*{1.22ex}\ensuremath{\rightarrow}\hspace*{1.22ex}\ensuremath{\gamma}
\end{minipage}}}


\medskip\noindent
  \begin{itemize}
  \item \texttt{f} operations always in parallel
  \item Two flavours: \texttt{map\_local\_fold} and
    \texttt{map\_remote\_fold}
  \item More parallelism when fold is associative and commutative
  \end{itemize}
  

\medskip\noindent
{\definecolor{tmpcolor}{rgb}{0.80,0.80,1.00}\colorbox{tmpcolor}{\begin{minipage}{\textwidth}\ttfamily\parindent 0pt

\noindent\hspace*{0.00em}{\color{blue}val}\hspace*{1.22ex}map\_{}fold\_{}ac,\hspace*{1.22ex}map\_{}fold\_{}a\symbol{58}~\linebreak
\noindent\hspace*{1.00em}f\symbol{58}\hspace*{1.22ex}(\ensuremath{\alpha}\hspace*{1.22ex}\ensuremath{\rightarrow}\hspace*{1.22ex}\ensuremath{\beta})\hspace*{1.22ex}\ensuremath{\rightarrow}\hspace*{1.22ex}~\linebreak
\noindent\hspace*{1.00em}fold\symbol{58}\hspace*{1.22ex}(\ensuremath{\beta}\hspace*{1.22ex}\ensuremath{\rightarrow}\hspace*{1.22ex}\ensuremath{\beta}\hspace*{1.22ex}\ensuremath{\rightarrow}\hspace*{1.22ex}\ensuremath{\beta})\hspace*{1.22ex}\ensuremath{\rightarrow}\hspace*{1.22ex}\ensuremath{\beta}\hspace*{1.22ex}\ensuremath{\rightarrow}\hspace*{1.22ex}\ensuremath{\alpha}\hspace*{1.22ex}list\hspace*{1.22ex}\ensuremath{\rightarrow}\hspace*{1.22ex}\ensuremath{\beta}
\end{minipage}}}


\medskip\noindent
\end{frame}

%% -------------------------------------------------------------------- %%

\begin{frame}\frametitle {Low-Level API}
  \begin{itemize}
  \item User can interact with the execution of the distributed
    computation


\bigskip
  \item Examples:
    \begin{itemize}
    \item Monitoring applications: observation of consumption of
      resources, etc
    \item Interactive programs
    \end{itemize}
  \end{itemize}
\end{frame}

%% -------------------------------------------------------------------- %%

\begin{frame}\frametitle {Low-Level API}
\begin{itemize}
\item \textsf{{\color{blue}type}\hspace*{1.22ex}(\ensuremath{\alpha},\hspace*{1.22ex}\ensuremath{\gamma})\hspace*{1.22ex}computation}
\item creation
\end{itemize}
  

\medskip\noindent
{\definecolor{tmpcolor}{rgb}{0.80,0.80,1.00}\colorbox{tmpcolor}{\begin{minipage}{\textwidth}\ttfamily\parindent 0pt

\noindent\hspace*{0.00em}{\color{blue}val}\hspace*{1.22ex}create\symbol{58}\hspace*{1.22ex}worker\symbol{58}\hspace*{1.22ex}(\ensuremath{\alpha}\hspace*{1.22ex}\ensuremath{\rightarrow}\hspace*{1.22ex}\ensuremath{\beta})\hspace*{1.22ex}\ensuremath{\rightarrow}\hspace*{1.22ex}~\linebreak
\noindent\hspace*{7.00em}master\symbol{58}(\ensuremath{\alpha}\hspace*{1.22ex}\ensuremath{\times}\hspace*{1.22ex}\ensuremath{\gamma}\hspace*{1.22ex}\ensuremath{\rightarrow}\hspace*{1.22ex}\ensuremath{\beta}\hspace*{1.22ex}\ensuremath{\rightarrow}\hspace*{1.22ex}(\ensuremath{\alpha}\hspace*{1.22ex}\ensuremath{\times}\hspace*{1.22ex}\ensuremath{\gamma})list)\hspace*{1.22ex}\ensuremath{\rightarrow}~\linebreak
\noindent\hspace*{7.00em}(\ensuremath{\alpha}\hspace*{1.22ex}\ensuremath{\times}\hspace*{1.22ex}\ensuremath{\gamma})\hspace*{1.22ex}computation
\end{minipage}}}


\medskip\noindent
  \begin{itemize}
  \item adding new tasks
  \end{itemize}
  

\medskip\noindent
{\definecolor{tmpcolor}{rgb}{0.80,0.80,1.00}\colorbox{tmpcolor}{\begin{minipage}{\textwidth}\ttfamily\parindent 0pt

\noindent\hspace*{0.00em}{\color{blue}val}\hspace*{1.22ex}add\_{}task\symbol{58}\hspace*{1.22ex}(\ensuremath{\alpha}\hspace*{1.22ex}\ensuremath{\times}\hspace*{1.22ex}\ensuremath{\gamma})\hspace*{1.22ex}computation\hspace*{1.22ex}\ensuremath{\rightarrow}\hspace*{1.22ex}\ensuremath{\alpha}\hspace*{1.22ex}\ensuremath{\times}\hspace*{1.22ex}\ensuremath{\gamma}\hspace*{1.22ex}\ensuremath{\rightarrow}\hspace*{1.22ex}unit
\end{minipage}}}


\medskip\noindent
  \begin{itemize}
  \item performing one step of the computation
  \end{itemize}


\medskip\noindent
{\definecolor{tmpcolor}{rgb}{0.80,0.80,1.00}\colorbox{tmpcolor}{\begin{minipage}{\textwidth}\ttfamily\parindent 0pt

\noindent\hspace*{0.00em}{\color{blue}val}\hspace*{1.22ex}one\_{}step\symbol{58}\hspace*{1.22ex}(\ensuremath{\alpha}\hspace*{1.22ex}\ensuremath{\times}\hspace*{1.22ex}\ensuremath{\gamma})\hspace*{1.22ex}computation\hspace*{1.22ex}\ensuremath{\rightarrow}\hspace*{1.22ex}unit
\end{minipage}}}


\medskip\noindent
\begin{itemize}
\item etc.
\end{itemize}
\end{frame}

%% -------------------------------------------------------------------- %%

\begin{frame}\frametitle{Outline}
    \begin{itemize}
    \item API
      \begin{itemize}
      \item general-purpose \texttt{compute} function
      \item high-level: map/fold operations
      \item low-level: micro-step computations
      \end{itemize}


\bigskip
    \item \emph{Deployment Scenarios}
      \begin{itemize}
      \item Sequential
      \item Cores
      \item Network
      \end{itemize}
    \end{itemize}

\end{frame}
% \begin{frame}\frametitle {Deployment Scenarios}
%   \begin{itemize}
%   \item Purely sequential execution: for debugging and comparison
%     purposes
%   \item Several cores on the same machine
%   \item Computation distributed over a network of machines
%   \end{itemize}
  
% \end{frame}

%% -------------------------------------------------------------------- %%

\begin{frame}\frametitle {Cores Implementation}
  Uses \texttt{Unix.fork} (no control over scheduling)

  

\medskip\noindent
{\definecolor{tmpcolor}{rgb}{0.80,0.80,1.00}\colorbox{tmpcolor}{\begin{minipage}{\textwidth}\ttfamily\parindent 0pt

\noindent\hspace*{1.00em}{\color{blue}open}\hspace*{1.22ex}Cores~\linebreak
\noindent\hspace*{1.00em}{\color{blue}let}\hspace*{1.22ex}()\hspace*{1.22ex}=\hspace*{1.22ex}set\_{}number\_{}of\_{}cores\hspace*{1.22ex}4
\end{minipage}}}


\medskip\noindent

  \begin{center}
    \includegraphics{master_workers_cores.mps}
  \end{center}

  master maintains a queue of pending tasks
  
\end{frame}

%% -------------------------------------------------------------------- %%


\begin{frame}\frametitle {Network Implementation}
  based on
  \begin{itemize}
  \item TCP/IP client/server architecture
  \item Ocaml's marshaling capabilities
  \end{itemize}
  
  \vfill
  marshaling considerations
  \begin{enumerate}
  \item \emph{same binary}: we can marshal closures
  \item \emph{same version of Ocaml}: we can only marshal values
  \item \emph{otherwise}: we can only marshal strings
  \end{enumerate}
  
\end{frame}

%% -------------------------------------------------------------------- %%

\begin{frame}\frametitle {Three Implementations of Network}
  
\emph{Same binary}


\medskip\noindent
{\definecolor{tmpcolor}{rgb}{0.80,0.80,1.00}\colorbox{tmpcolor}{\begin{minipage}{\textwidth}\ttfamily\parindent 0pt

\noindent\hspace*{0.00em}{\color{blue}val}\hspace*{1.22ex}compute\hspace*{1.22ex}\symbol{58}\hspace*{1.22ex}{\color{red}(*\hspace*{1.22ex}same\hspace*{1.22ex}as\hspace*{1.22ex}before\hspace*{1.22ex}*)}\hspace*{1.22ex}...
\end{minipage}}}


\medskip\noindent



\bigskip
\emph{Same version of Caml}
    

\medskip\noindent
{\definecolor{tmpcolor}{rgb}{0.80,0.80,1.00}\colorbox{tmpcolor}{\begin{minipage}{\textwidth}\ttfamily\parindent 0pt

\noindent\hspace*{0.00em}{\color{blue}val}\hspace*{1.22ex}Worker.compute\hspace*{1.22ex}\symbol{58}\hspace*{1.22ex}(\ensuremath{\alpha}\hspace*{1.22ex}\ensuremath{\rightarrow}\hspace*{1.22ex}\ensuremath{\beta})\hspace*{1.22ex}\ensuremath{\rightarrow}\hspace*{1.22ex}unit~\linebreak
\noindent\hspace*{0.00em}{\color{blue}val}\hspace*{1.22ex}Master.compute\hspace*{1.22ex}\symbol{58}\hspace*{1.22ex}~\linebreak
\noindent\hspace*{1.00em}(\ensuremath{\alpha}\hspace*{1.22ex}\ensuremath{\times}\hspace*{1.22ex}\ensuremath{\gamma}\hspace*{1.22ex}\ensuremath{\rightarrow}\hspace*{1.22ex}\ensuremath{\beta}\hspace*{1.22ex}\ensuremath{\rightarrow}\hspace*{1.22ex}(\ensuremath{\alpha}\hspace*{1.22ex}\ensuremath{\times}\hspace*{1.22ex}\ensuremath{\gamma})\hspace*{1.22ex}list)\hspace*{1.22ex}\ensuremath{\rightarrow}~\linebreak
\noindent\hspace*{1.00em}(\ensuremath{\alpha}\hspace*{1.22ex}\ensuremath{\times}\hspace*{1.22ex}\ensuremath{\gamma})\hspace*{1.22ex}list\hspace*{1.22ex}\ensuremath{\rightarrow}\hspace*{1.22ex}unit
\end{minipage}}}


\medskip\noindent



\bigskip
\emph{Otherwise}
    

\medskip\noindent
{\definecolor{tmpcolor}{rgb}{0.80,0.80,1.00}\colorbox{tmpcolor}{\begin{minipage}{\textwidth}\ttfamily\parindent 0pt

\noindent\hspace*{0.00em}{\color{blue}val}\hspace*{1.22ex}Worker.compute\hspace*{1.22ex}\symbol{58}\hspace*{1.22ex}(string\hspace*{1.22ex}\ensuremath{\rightarrow}\hspace*{1.22ex}string)\hspace*{1.22ex}\ensuremath{\rightarrow}\hspace*{1.22ex}unit~\linebreak
\noindent\hspace*{0.00em}{\color{blue}val}\hspace*{1.22ex}Master.compute\hspace*{1.22ex}\symbol{58}\hspace*{1.22ex}~\linebreak
\noindent\hspace*{1.00em}(string\hspace*{1.22ex}\ensuremath{\times}\hspace*{1.22ex}\ensuremath{\gamma}\hspace*{1.22ex}\ensuremath{\rightarrow}\hspace*{1.22ex}string\hspace*{1.22ex}\ensuremath{\rightarrow}\hspace*{1.22ex}(string\hspace*{1.22ex}\ensuremath{\times}\hspace*{1.22ex}\ensuremath{\gamma})\hspace*{1.22ex}list)\hspace*{1.22ex}\ensuremath{\rightarrow}~\linebreak
\noindent\hspace*{1.00em}(string\hspace*{1.22ex}\ensuremath{\times}\hspace*{1.22ex}\ensuremath{\gamma})\hspace*{1.22ex}list\hspace*{1.22ex}\ensuremath{\rightarrow}\hspace*{1.22ex}unit
\end{minipage}}}


\medskip\noindent

\end{frame}

%% -------------------------------------------------------------------- %%

\begin{frame}\frametitle {Network Implementation Details}
  

\medskip\noindent
{\definecolor{tmpcolor}{rgb}{0.80,0.80,1.00}\colorbox{tmpcolor}{\begin{minipage}{\textwidth}\ttfamily\parindent 0pt

\noindent\hspace*{1.00em}{\color{blue}open}\hspace*{1.22ex}Network~\linebreak
\noindent\hspace*{1.00em}{\color{blue}let}\hspace*{1.22ex}()\hspace*{1.22ex}=\hspace*{1.22ex}declare\_{}workers\hspace*{1.22ex}\symbol{126}n\symbol{58}3\hspace*{1.22ex}"moloch"~\linebreak
\noindent\hspace*{1.00em}{\color{blue}let}\hspace*{1.22ex}()\hspace*{1.22ex}=\hspace*{1.22ex}declare\_{}workers\hspace*{1.22ex}\symbol{126}n\symbol{58}2\hspace*{1.22ex}"orcus"
\end{minipage}}}


\medskip\noindent

  \begin{center}
    \includegraphics{master_workers_network.mps}
  \end{center}
  Each worker behaves as a server, the master being the client
  
\end{frame}

%% -------------------------------------------------------------------- %%

\begin{frame}\frametitle {Underneath: Protocol}
  \only<1>{Master sends a task to a worker}
  \only<2>{Worker computes and sends back a result}
  \only<3-4>{Master and workers exchange \emph{ping}/\emph{pong} messages}
  \only<5>{Master sends another task to a worker}
  \only<6>{In case of a disconnection...}
  \only<7>{The task is \emph{rescheduled} to another worker}
  \only<8>{Whenever one completes...}
  \only<9>{The other one is stopped}
  \only<10>{The master is notified when a computation fails}
  \only<11>{At the very end, the master may ask the workers to stop}
  \vfill
  \begin{center}
    \only<1,5>{\includegraphics{master_workers_assign.mps}}
    \only<2>{\includegraphics{master_workers_completed.mps}}
    \only<3>{\includegraphics{master_workers_ping.mps}}
    \only<4>{\includegraphics{master_workers_pong.mps}}
    \only<6>{\includegraphics{master_workers_disconnection.mps}}
    \only<7>{\includegraphics{master_workers_assign_2.mps}}
    \only<8>{\includegraphics{master_workers_completed_2.mps}}
    \only<9>{\includegraphics{master_workers_kill.mps}}
    \only<10>{\includegraphics{master_workers_aborted.mps}}
    \only<11>{\includegraphics{master_workers_stop.mps}}
  \end{center}
  \begin{center}
    \only<1>{\textsf{Assign(42,\hspace*{1.22ex}f,\hspace*{1.22ex}a)}}
    \only<2>{\textsf{Completed\hspace*{1.22ex}(42,\hspace*{1.22ex}b)}}
    \only<3>{\textsf{Ping}}
    \only<4>{\textsf{Pong}}
    \only<5>{\textsf{Assign(43,\hspace*{1.22ex}f,\hspace*{1.22ex}a)}}
    \only<6>{\textsf{}} % disconnection
    \only<7>{\textsf{Assign(43,\hspace*{1.22ex}f,\hspace*{1.22ex}a)}}
    \only<8>{\textsf{Completed\hspace*{1.22ex}(43,\hspace*{1.22ex}b)}}
    \only<9>{\textsf{Kill\hspace*{1.22ex}43}}
    \only<10>{\textsf{Aborted\hspace*{1.22ex}44}}
    \only<11>{\textsf{Stop}}
  \end{center}
\end{frame}

%% -------------------------------------------------------------------- %%


\begin{frame}\frametitle {Fault Tolerance}

Master knows the state of each worker through ping/pong messages

  \begin{center}
    \includegraphics{state.mps}
  \end{center}

  
\end{frame}

%% -------------------------------------------------------------------- %%


\begin{frame}\frametitle {Experimental Results}
  Motivating example
  \begin{itemize}
  \item 80 verification conditions / 4 provers = 320 tasks
  \item network of 3 machines (4, 8 and 8 cores)
  \end{itemize}

  \begin{itemize}
  \item sequential computation: \emph{$>$ 6 hours}
  \item with Functory: \emph{22.5 minutes}
  \item speedup ratio = \emph{16} (optimal is 20)
  \end{itemize}


\bigskip
  More experimental results in the paper: \par N-queens, Mandelbrot set,
  matrix multiplication
\end{frame}

%% -------------------------------------------------------------------- %%

\begin{frame}\frametitle {Related Work}
  Distributed Functional Languages (DFL)
  \begin{itemize}
  \item Jo\&Caml - rich communication primitives, no easy features
    for fault-tolerance
  \item ML5 - code mobility, marshalling, etc, but no
    easy features for fault-tolerance
  \item Glasgow Distributed Haskell - features for fault-tolerance,
    error detection/recovery
  \end{itemize}
  Libraries for existing functional languages
  \begin{itemize}
  \item Plasma MR - OCaml implementation of Google's Map/Reduce on
    Plasma FS
  \item iTask - a library for 'Clean' language for distributed workflow management
  \end{itemize}

Our library could be implemented using any of the DFLs, but our goals
are:
\begin{itemize}
\item Provide users of an existing language, user-friendly APIs
\item Provide robust fault-tolerance mechanism 
\item Relieve the users from coding messy details
\end{itemize}
  
\end{frame}

%% -------------------------------------------------------------------- %%

\begin{frame}\frametitle {What's next}
  \begin{itemize}
  \item Better Scheduling
  \item Real-time visualization - OcamlViz?
  \item Any gains from Plasma-FS - data locality?
  \end{itemize}
\end{frame}

%% -------------------------------------------------------------------- %%

\begin{frame}\frametitle {Thanks}


\bigskip
Check out:
\begin{center}
 \textcolor{blue}{\url{http://functory.lri.fr/}}  
\end{center}

Feedback, comments welcome!

\end{frame}

%% -------------------------------------------------------------------- %%
\end{document}






















































































\bigskip\bigskip\bigskip
\begin{frame}\frametitle{The Talk}


\bigskip
\begin{bbox}
\begin{alltt}\intt main ()\\
\{\\
  \- \- \intt x = 0;\\
  \- \- \intt y = 1;\\
  \- \- \while (x < 100) \{\\
  \- \- \- \- x = x + y;\\
  \- \- \- \- y = x;\\
  \- \- \}\\
  \- \- \asrt (x < 200);\\
  \- \- \ret 0;\\
\}
\end{alltt}%
\end{bbox}
\pause

Can we automatically: \\ 
\begin{itemize}
  \itemsep=0.3cm
\item Generate a loop invariant ?
\item Generate a loop invariant that helps in proving the assertion ?

\end{itemize}

\end{frame}

%% -------------------------------------------------------------------- %%

\begin{frame}\frametitle {Abstraction - Intuition}
  \centering
  \includegraphics[scale=0.3]{Diagram1_new1.png}
\end{frame}

%% -------------------------------------------------------------------- %%

\begin{frame}\frametitle {Abstraction - Intuition}
  \centering
  \includegraphics[scale=0.3]{Diagram2_new1.png}
\end{frame}

%% -------------------------------------------------------------------- %%

\begin{frame}\frametitle {Predicate Abstraction}
The Problem: \\ \- \\ Given a concrete program and a set of predicates, track
the behaviour of the predicates corresponding to the behaviour of the
concrete program \\ \- \\
\pause
Ingredients:
  \begin{itemize}
    \itemsep=0.3cm
  \item Concrete program: $R(\vec{x}, \vec{x}')$
  \item A set of predicates: \{$\vec{P} \iff \alpha (\vec{x})$, $\vec{P}' \iff
    \alpha (\vec{x'})$ \} 
  \item Behaviour of the predicates w.r.t the concrete program:
    $$
    \exists. \vec{x}, \vec{x}'.\ R(\vec{x}, \vec{x}') \land \vec{P} =
    \alpha (\vec{x}) \land \vec{P}' = \alpha (\vec{x'})
    $$
  \item Existential quantification - solved using a SMT solver
  \item Abstract interpretation with Abstract domain =
      vector of Booleans
  \end{itemize}

\end{frame}

%% -------------------------------------------------------------------- %%

\begin{frame}\frametitle{Predicate Abstraction - Example}
  \begin{columns}
    \begin{column}{0.65\textwidth}
      \begin{itemize}
        \itemsep=0.3cm
      \item Concrete initial states : \\ $I_c(x, y) = (x=0 \land y=1)$
      \item Concrete transitions : \\ $R_c(x, x', y, y') = \newline (x<100) \implies
        (x'=x+y \land y'=x)$
      \item Predicate : $\vec{P} = \{(b_0: (x<200))\}$
      \item<2-> Abstract initial states: \\
        $I_a(b_0) = \exists x,y. (x=0 \land y=1) \land (b_0 \iff x<200) $
      \item<2-> Abstract transitions: \\
        $R_a(b_0, b_0') = \exists x,y,x',y'. \newline ((x<100) \implies (x'=x+y
        \land y'=x) \land \newline
        (b_0 \iff x<200) \land (b_0' \iff x'<200)$
      \end{itemize}
    \end{column}
    \begin{column}{0.35\textwidth}
      \uncover<1->{
      \begin{bbox}
        \begin{alltt}          \intt x = 0;\\
          \intt y = 1;\\
          \while (x < 100) \{\\
          \- \- x = x + y;\\
          \- \- y = x;\\
          \}\\
        \end{alltt}%
      \end{bbox}}
      \newline
      \uncover<2->{
      \ts{abstract initial state:} \\
      \includegraphics[scale=0.25]{init.png} \\
      \ts{abstract transitions:} \\
      \includegraphics[scale=0.25]{trans.png}
    }
    \end{column}
  \end{columns}
\end{frame}

%% -------------------------------------------------------------------- %%

\begin{frame}\frametitle {Abstraction at Program Points}
  \begin{itemize}
    \itemsep=0.4cm
  \item Compute CFG of the concrete program
  \item Choose a set of predicates
  \item Compute abstractions at each program point in the CFG
    \pause
  \item Intuitively,
    \pause
    \begin{itemize}
      \itemsep=0.3cm
    \item The predicate values at each program point is an invariant
      at the program point 
      \pause
    \item Concretization of the predicates correspond to
      the invariant of the concrete program
    \end{itemize}
  \end{itemize}
\end{frame}

%% -------------------------------------------------------------------- %%

\begin{frame}\frametitle {Let's begin}
  \begin{columns}
    \begin{column}{0.5\textwidth}
      \begin{bbox}
        \begin{alltt}          \intt main ()\\
          \{\\
          \- \- \intt x = 0;\\
          \- \- \intt y = 1;\\
          \- \- \while (x < 100) \{\\
          \- \- \- \- x = x + y;\\
          \- \- \- \- y = x;\\
          \- \- \}\\
          \- \- \ret 0;\\
          \}
        \end{alltt}%
      \end{bbox}
      \begin{itemize}
        \itemsep=0.3cm
      \item compute the CFG
      \item abstract it at program points
      \end{itemize}
    \end{column}
    
    \begin{column}{0.5\textwidth}
      \includegraphics[scale=0.3]{c.pdf}  
    \end{column}
  \end{columns}

\end{frame}

%% -------------------------------------------------------------------- %%

\begin{frame}\frametitle{Abstract Transitions}
\begin{minipage}{6.5cm}
  \begin{itemize}
\itemsep=0.4cm
  \item Consider the first concrete transition
  \item Assume predicates: \\ 
    \begin{bbox}
      {\scriptsize{
          $b_0: (x < 200), b_1: (x < 100), b_2: (x + y < 200)$      
        }}
    \end{bbox}
  \item<2-> Compute abstraction w.r.t the predicates:

    \begin{bbox}
      {\scriptsize{
          $
          \exists x_0, x_1, y_0, y_1.\ \newline  
          (x_1=0 \land y_1=1)\ \land \newline
          (b_0 \iff (x_0 < 200) \ \land \ (b_0' \iff (x_1 < 200) \ \land \newline 
          (b_1 \iff (x_0 < 100) \ \land \ (b_1' \iff (x_1 < 100) \ \land  \newline
          (b_2 \iff (x_0 + y_0 < 200) \ \land \ (b_2' \iff (x_1 + y_1 < 200)
          $
        }}
    \end{bbox}
    
  \item<3-> solve the above using alt-ergo
  \item<3-> solutions: \\ 
    \begin{bbox}
      {\scriptsize{
        $(b_0' = true), (b_1' = true), (b_2' = true)$      
      }}
    \end{bbox}
  \end{itemize}

\end{minipage}
\hspace{1cm}
\begin{minipage}{3cm}
\alt<3->
  { \includegraphics[scale=0.3]{abs-ex-1.pdf} }
  { \includegraphics[scale=0.3]{conc-1.pdf} }
\end{minipage}
\end{frame}

%% -------------------------------------------------------------------- %%

\begin{frame}\frametitle{Abstract Program - Loop Invariant?}
\begin{columns}
  \begin{column}{0.65\textwidth}
    \begin{itemize}
      \itemsep=0.3cm
    \item<1-> Final abstract system
      \item<2-> Collect the predicates at the loop head (here, 2)
      \item<3-> $(b_0 \land b_1 \land b_2)\  \lor\ (\lnot b_0 \land b_1
        \land \lnot b_2)$ \\ is a loop invariant
      \item<4-> which is: \\
        $ 
        ((x<100) \land (x<200) \land (x+y<200)) \ \lor \newline
        (\lnot (x<100) \land (x<200) \land \lnot (x+y<200))
        $
      \end{itemize}
  \end{column}
\begin{column}{0.35\textwidth}

  \alt<2->{\includegraphics[scale=0.3]{abs-final2.png}}{\includegraphics[scale=0.3]{abs-final.png}}
\end{column}
\end{columns}
\end{frame}

%% -------------------------------------------------------------------- %%

\begin{frame}\frametitle{Demo}

\end{frame}

%% -------------------------------------------------------------------- %%

\begin{frame}\frametitle{Part - II}
  \begin{itemize}
    \itemsep=0.4cm
  \item What predicates are interesting?
  \item How to discover new predicates?
  \item How did we collect predicates?
  \end{itemize}

\end{frame}

%% -------------------------------------------------------------------- %%

\begin{frame}\frametitle{Predicate Discovery}
  When proving properties about programs, we need \\ \- \\
  \begin{itemize}
    \itemsep=0.4cm
  \item Loop invariants that will help in proving the property
  \item Predicates (interesting) to generate such an invariant
  \item Interesting predicates = useful information about the program
    w.r.t property
  \item To be able to construct abstractions incrementally
  \end{itemize}
\end{frame}

%% -------------------------------------------------------------------- %%

\begin{frame}\frametitle{Abstraction - Recap}
  \centering
  \includegraphics[scale=0.3]{Diagram2_new1.png}
\end{frame}

\begin{frame}{Refinement - Intuition}
  \centering
  \includegraphics[scale=0.3]{Diagram3.png}
\end{frame}

%% -------------------------------------------------------------------- %%

\begin{frame}\frametitle{Predicate Discovery: WP Computation}
  Weakest Precondition Computation \\
  \begin{itemize}
    \itemsep=0.3cm
  \item Given a property P and a loop body $B$
  \item Compute $P_{new} = WP(P, B)$ 
  \item Add the atomic predicates in $P_{new}$ as new predicates
  \item Intuitively, $P_{new}$ is some information relevant to the property
    and driven by the statements in $B$
  \end{itemize}
  
\end{frame}

%% -------------------------------------------------------------------- %%

\begin{frame}\frametitle{Predicate Discovery: Other Techniques}
  Craig Interpolants\\ 
  \begin{itemize}
    \itemsep=0.3cm
  \item Given a pair of mutually inconsistent formulas $\phi_1$ and
    $\phi_2$, there is an interpolant $\psi$, with the property
    \begin{itemize}
      \itemsep=0.2cm
    \item $\phi_1 \rightarrow \psi$
    \item $\psi \land \phi_2$ is unsatisfiable
    \item $\psi$ refers only to the variables common to both $\phi_1$
      and $\phi_2$
  \end{itemize}
\item Intuitively, the interpolant $\psi$ extracts the relevant parts
  of $\phi_1$.
\item If a set of first-order formulas is inconsistent, an interpolant
  always exists and can be computed from the proof of unsatisfiability
  of the candidate formulas.
  
\end{itemize}
\end{frame}

%% -------------------------------------------------------------------- %%

\begin{frame}\frametitle{Predicate Discovery}
  \begin{columns}
    \begin{column}{0.6\textwidth}
      In our example, \\ \- \\
      Suppose we want to prove the assertion $(x < 200)$ \\ \-
      \newline
      \uncover<2->{
      Predicates:}
      \begin{itemize}
      \item<2-> Guards
      \item<2-> The property to be proved
      \item<3-> WP of the property w.r.t loop body $B$
        $$
        WP (x<200, B) =\hl{$x+y < 200$} 
        $$
      \end{itemize}
    \end{column}
    \begin{column}{0.4\textwidth}
      \alt<2->{
      \begin{bbox}
        \begin{alltt}          \intt main ()\\
          \{\\
          \- \- \intt x = 0;\\
          \- \- \intt y = 1;\\
          \- \- \while \hl{(x < 100)} \{\\
          \- \- \- \- x = x + y;\\
          \- \- \- \- y = x;\\
          \- \- \}\\
          \- \- \asrt \hl{(x < 200)};\\
          \- \- \ret 0;\\
          \}
        \end{alltt}%
      \end{bbox}}
    {
      \begin{bbox}
        \begin{alltt}          \intt main ()\\
          \{\\
          \- \- \intt x = 0;\\
          \- \- \intt y = 1;\\
          \- \- \while (x < 100) \{\\
          \- \- \- \- x = x + y;\\
          \- \- \- \- y = x;\\
          \- \- \}\\
          \- \- \asrt (x < 200);\\
          \- \- \ret 0;\\
          \}
        \end{alltt}%
      \end{bbox}
    }
    \end{column}
  \end{columns}
  
\end{frame}

%% -------------------------------------------------------------------- %%

\begin{frame}\frametitle{Other Tools/Techniques}
  \begin{itemize}
    \itemsep=0.3cm
  \item Most abstraction techniques abstract the whole program
  \item SLAM abstracts at each program statement but does not
    infer invariants
  \item Most abstraction techniques also implement refinement
    framework \\ (BLAST, CBMC, NuSMT)
  \end{itemize}
\end{frame}

%% -------------------------------------------------------------------- %%

\begin{frame}\frametitle{BLAST - Lazy Abstraction}
  BLAST - Lazy Abstraction
  \begin{itemize}
  \item Most generic predicate is used to start with
  \item Iteratively refined 
  \item Abstraction is coarse - so more iterations
  \end{itemize}
  
\end{frame}

%% -------------------------------------------------------------------- %%

\begin{frame}\frametitle{NuSMT: SMT + BDD}
  NuSMT - SMT + BDD-based Abstraction
  \begin{itemize}
  \item Tight integration of BDD enumeration and SMT reasoning
  \item CEGAR loop with predicate discovery techniques
  \item Computes precise abstraction - less iterations
  \item No invariant generation
  \end{itemize}
\end{frame}

%% -------------------------------------------------------------------- %%

\begin{frame}\frametitle{SAT-based Abstraction}
CBMC - SAT-based Abstraction
  \begin{itemize}
  \item SAT based abstraction
    \item Encoding in SAT an issue 
    \item Imprecise abstractions - more iterations of CEGAR loop
    \end{itemize}
\end{frame}

%% -------------------------------------------------------------------- %%

\begin{frame}\frametitle{Directions for Future Work}
  \begin{itemize}
    \itemsep=0.4cm
  \item Use WP computations to discover predicates
  \item Generating quantified invariants automatically
  \item Iterative abstraction techniques - model-checker?
  \item Use information from abstract interpretation techniques
    in Frama-C \\(APRON library?)
  \end{itemize}
  
\end{frame}



\bigskip\bigskip
%% -------------------------------------------------------------------- %%

\begin{frame}\frametitle{\begin{center} Merci! \end{center}}
  
\end{frame}

%% -------------------------------------------------------------------- %%



\bigskip
\end{document}


\bigskip
%%% Local Variables: 
%%% mode: latex
%%% TeX-master: t
%%% End: 

