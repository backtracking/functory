% \documentclass[xcolor=dvipsnames]{beamer}
% % \documentclass[handout]{beamer}
% \usecolortheme[named=Maroon]{structure} 
% \usetheme{Boadilla}
% \setbeamertemplate{beaver}
\documentclass[xcolor=dvipsnames]{beamer} 
\usecolortheme[named=MidnightBlue]{structure} 
\usetheme[height=7mm]{Rochester} 
\usepackage{fontenc}
\renewcommand*\familydefault{\sfdefault}
\usepackage{times}
% \usepackage[latin1]{inputenc}
\usepackage{alltt}
\renewcommand{\iff}{\leftrightarrow}
\newcommand{\tr}{\textcolor{red}}
\newcommand{\tb}{\textcolor{MidnightBlue}}
\newcommand{\intt}{\textcolor{blue}{\textbf{int }}}
\newcommand{\while}{\textcolor{blue}{\textbf{while }}}
\newcommand{\ret}{\textcolor{blue}{\textbf{return }}}
\newcommand{\asrt}{\textcolor{blue}{\textbf{//@assert }}}
\newcommand{\ts}[1]{\scriptsize{#1}}
\newcommand{\hl}[1]{\colorbox{yellow}{#1}}
%\usepackage[french]{babel}

\beamertemplatenavigationsymbolsempty


\definecolor{MyGray}{rgb}{0.86,0.87,0.98}
\makeatletter\newenvironment{bbox}{%
   \begin{lrbox}
     {\@tempboxa}\begin{minipage}
       {\columnwidth}}
     {\end{minipage}
   \end{lrbox}%
   \colorbox{MyGray}{\usebox{\@tempboxa}}
}\makeatother
% \definecolor{MyGray}{rgb}{0.96,0.97,0.98}


\definecolor{kwblue}{rgb}{0.67,0.12,0.92}
\definecolor{ceruleanblue}{rgb}{0, 0.48, 0.65}
\definecolor{lightpink}{rgb}{1., 0.71, 0.75}
\definecolor{lightblue}{rgb}{0.8,0.8,1}
\definecolor{lightred}{rgb}{1,0.8,0.8}

\let\emph\alert
\newcommand{\coeurs}{c\oe urs}

\begin{document}

%% -------------------------------------------------------------------- %%

\title{FUNCTORY \\ A Distributed Computing Library \\ for Objective Caml}

\author[Kalyan]{Jean-Christophe Filli\^{a}tre \& K. Kalyanasundaram\\
  \- \\ 
  CNRS / INRIA Saclay -- \^{I}le-de-France
\\
\vspace{0.5cm}
TFP 2011, Madrid, Spain
}


\date {}


\begin{frame}
  \titlepage
  \begin{center}
    \includegraphics[scale=0.11]{cnrs-logo2.jpg}
    \hspace{4mm}
    \includegraphics[scale=0.09]{lrilogo.jpg}
    \hspace{4mm}
    \includegraphics[scale=0.09]{upsudlogo.jpg}
    \hspace{4mm}
    \includegraphics[scale=0.4]{inria-saclay.jpeg}
    % \pgfimage[height=6mm]{saclay}
  \end{center}
\end{frame}

%% -------------------------------------------------------------------- %%

\begin{frame}\frametitle{Motivation}
  \begin{itemize}
  \item In our team, we do deductive program verification
  \item Generates numerous verification conditions
  \item Discharged by various automated provers
  \item Typically takes \emph{hours} to complete
  \end{itemize}
  \vspace{1cm}
  \begin{itemize}
  \item Some multi-core machines at our disposal
  \item How to make the best possible use of them?
  \end{itemize}
\end{frame}

%% -------------------------------------------------------------------- %%

\begin{frame}\frametitle {A Distributed Computing Library}
  Requirements
  \begin{itemize}
  \item Fault-tolerance
  \item User-friendly API
  \item In our favorite programming language (OCaml)
  \end{itemize}


Also
  \begin{itemize}
  \item General purpose library
  \item Portability
  \end{itemize}
\end{frame}

%% -------------------------------------------------------------------- %%

\begin{frame}\frametitle {Basic Design}

Inspired by Google's Map/Reduce (OSDI 2004)

\begin{itemize}
\item Workers in parallel
\item Master
\end{itemize}

\begin{center}
  \includegraphics{master_workers_1.mps}  
\end{center}
\end{frame}

%% -------------------------------------------------------------------- %%

\begin{frame}\frametitle{Outline}
    \begin{itemize}
    \item API
      \begin{itemize}
      \item general-purpose \texttt{compute} function
      \item high-level: map/fold operations
      \item low-level: micro-step computations
      \end{itemize}


    \item Deployment Scenarios
      \begin{itemize}
      \item Sequential
      \item Cores
      \item Network (3 flavours)
      \end{itemize}


    \item Many libraries in one

    \end{itemize}

\end{frame}

%% -------------------------------------------------------------------- %%

\begin{frame}\frametitle{A General-Purpose \texttt{compute} Function}
  \begin{ocaml}
    val compute : 
    worker:('a -> 'b) -> 
    master:('a * 'c -> 'b -> ('a * 'c) list) -> 
    ('a * 'c) list -> 
    unit
  \end{ocaml}
  \begin{itemize}
  \item A task is of type \ocaml{'a * 'c}, its result of type
    \ocaml{'b}
  \item A completed task may in turn generate new tasks
  \item \texttt{compute} returns when there is no more task
  \end{itemize}
\end{frame}

%% -------------------------------------------------------------------- %%

\begin{frame}\frametitle {High-Level API}
  \begin{itemize}
  \item most common map/fold operations over lists
  \end{itemize}
  \begin{ocaml}
val map: f:('a -> 'b) -> 'a list -> 'b list
val map_fold: f:('a -> 'b) -> 
              fold:('c -> 'b -> 'c) -> 'c -> 'a list -> 'c
  \end{ocaml}
  \begin{itemize}
  \item \texttt{f} operations always in parallel
  \item Two flavours: \texttt{map\_local\_fold} and
    \texttt{map\_remote\_fold}
  \item More parallelism when fold is associative and commutative
  \end{itemize}
  \begin{ocaml}
val map_fold_ac, map_fold_a:
  f: ('a -> 'b) -> 
  fold: ('b -> 'b -> 'b) -> 'b -> 'a list -> 'b
  \end{ocaml}
\end{frame}

%% -------------------------------------------------------------------- %%

\begin{frame}\frametitle {Low-Level API}
  \begin{itemize}
  \item User can interact with the execution of the distributed
    computation


  \item Examples:
    \begin{itemize}
    \item Monitoring applications: observation of consumption of
      resources, etc
    \item Interactive programs
    \end{itemize}
  \end{itemize}
\end{frame}

%% -------------------------------------------------------------------- %%

\begin{frame}\frametitle {Low-Level API}
\begin{itemize}
\item \ocaml{type ('a, 'c) computation}
\item creation
\end{itemize}
  \begin{ocaml}
val create: worker: ('a -> 'b) -> 
              master:('a * 'c -> 'b -> ('a * 'c)list) ->
              ('a * 'c) computation
  \end{ocaml}
  \begin{itemize}
  \item adding new tasks
  \end{itemize}
  \begin{ocaml}
val add_task: ('a * 'c) computation -> 'a * 'c -> unit
  \end{ocaml}
  \begin{itemize}
  \item performing one step of the computation
  \end{itemize}
\begin{ocaml}
val one_step: ('a * 'c) computation -> unit
\end{ocaml}
\begin{itemize}
\item etc.
\end{itemize}
\end{frame}

%% -------------------------------------------------------------------- %%

\begin{frame}\frametitle{Outline}
    \begin{itemize}
    \item API
      \begin{itemize}
      \item general-purpose \texttt{compute} function
      \item high-level: map/fold operations
      \item low-level: micro-step computations
      \end{itemize}


    \item \emph{Deployment Scenarios}
      \begin{itemize}
      \item Sequential
      \item Cores
      \item Network
      \end{itemize}
    \end{itemize}

\end{frame}
% \begin{frame}\frametitle {Deployment Scenarios}
%   \begin{itemize}
%   \item Purely sequential execution: for debugging and comparison
%     purposes
%   \item Several cores on the same machine
%   \item Computation distributed over a network of machines
%   \end{itemize}
  
% \end{frame}

%% -------------------------------------------------------------------- %%

\begin{frame}\frametitle {Cores Implementation}
  Uses \texttt{Unix.fork} (no control over scheduling)

  \begin{ocaml}
  open Cores
  let () = set_number_of_cores 3
  \end{ocaml}

  \begin{center}
    \includegraphics{master_workers_cores.mps}
  \end{center}

  master maintains a queue of pending tasks
  
\end{frame}

%% -------------------------------------------------------------------- %%


\begin{frame}\frametitle {Network Implementation}
  based on
  \begin{itemize}
  \item TCP/IP client/server architecture
  \item Ocaml's marshaling capabilities
  \end{itemize}
  
  \vfill
  marshaling considerations
  \begin{enumerate}
  \item \emph{same binary}: we can marshal closures
  \item \emph{same version of Ocaml}: we can only marshal values
  \item \emph{otherwise}: we can only marshal strings
  \end{enumerate}
  
\end{frame}

%% -------------------------------------------------------------------- %%

\begin{frame}\frametitle {Three Implementations of Network}
  
\emph{Same binary}
\begin{ocaml}
val compute : (* same as before *) ...
\end{ocaml}



\emph{Same version of Caml}
    \begin{ocaml}
val Worker.compute : ('a -> 'b) -> unit
val Master.compute : 
  ('a * 'c -> 'b -> ('a * 'c) list) ->
  ('a * 'c) list -> unit
\end{ocaml}



\emph{Otherwise}
    \begin{ocaml}
val Worker.compute : (string -> string) -> unit
val Master.compute : 
  (string * 'c -> string -> (string * 'c) list) ->
  (string * 'c) list -> unit
\end{ocaml}

\end{frame}

%% -------------------------------------------------------------------- %%

\begin{frame}\frametitle {Network Implementation Details}
  \begin{ocaml}
  open Network
  let () = declare_workers ~n:3 "moloch"
  let () = declare_workers ~n:2 "orcus"
  \end{ocaml}

  \begin{center}
    \includegraphics{master_workers_network.mps}
  \end{center}
  Each worker behaves as a server, the master being the client
  
\end{frame}

%% -------------------------------------------------------------------- %%

\begin{frame}\frametitle {Protocol}
  \only<1>{Master sends a task to a worker}
  \only<2>{Worker computes and sends back a result}
  \only<3-4>{Master and workers exchange \emph{ping}/\emph{pong} messages}
  \only<5>{Master sends another task to a worker}
  \only<6>{In case of a disconnection...}
  \only<7>{The task is \emph{rescheduled} to another worker}
  \only<8>{Whenever one completes...}
  \only<9>{The other one is stopped}
  \only<10>{The master is notified when a computation fails}
  \only<11>{At the very end, the master may ask the workers to stop}
  \vfill
  \begin{center}
    \only<1,5>{\includegraphics{master_workers_assign.mps}}
    \only<2>{\includegraphics{master_workers_completed.mps}}
    \only<3>{\includegraphics{master_workers_ping.mps}}
    \only<4>{\includegraphics{master_workers_pong.mps}}
    \only<6>{\includegraphics{master_workers_disconnection.mps}}
    \only<7>{\includegraphics{master_workers_assign_2.mps}}
    \only<8>{\includegraphics{master_workers_completed_2.mps}}
    \only<9>{\includegraphics{master_workers_kill.mps}}
    \only<10>{\includegraphics{master_workers_aborted.mps}}
    \only<11>{\includegraphics{master_workers_stop.mps}}
  \end{center}
  \begin{center}
    \only<1>{\ocaml{Assign(42, f, a)}}
    \only<2>{\ocaml{Completed (42, b)}}
    \only<3>{\ocaml{Ping}}
    \only<4>{\ocaml{Pong}}
    \only<5>{\ocaml{Assign(43, f, a)}}
    \only<6>{\ocaml{}} % disconnection
    \only<7>{\ocaml{Assign(43, f, a)}}
    \only<8>{\ocaml{Completed (43, b)}}
    \only<9>{\ocaml{Kill 42}}
    \only<10>{\ocaml{Aborted 42}}
    \only<11>{\ocaml{Stop}}
  \end{center}
\end{frame}

%% -------------------------------------------------------------------- %%


\begin{frame}\frametitle {Fault Tolerance}

Master knows the state of each worker through ping/pong messages

  \begin{center}
    \includegraphics{state.mps}
  \end{center}

  \begin{itemize}
  \item Time-out values for deciding the status of the worker
  \end{itemize}
  
\end{frame}

%% -------------------------------------------------------------------- %%


\begin{frame}\frametitle {Experimental Results}
  Motivating example
  \begin{itemize}
  \item 80 verification conditions / 4 provers = 320 tasks
  \item network of 3 machines (4, 8 and 8 cores)
  \end{itemize}

  \begin{itemize}
  \item sequential computation: \emph{$>$ 6 hours}
  \item with Functory: \emph{22.5 minutes}
  \item speedup ratio = \emph{16} (optimal is 20)
  \end{itemize}


  More experimental results in the paper: \par N-queens, Mandelbrot set,
  matrix multiplication
\end{frame}

%% -------------------------------------------------------------------- %%

\begin{frame}\frametitle {Related Work}
  Distributed Functional Languages (DFL)
  \begin{itemize}
  \item Jo\&Caml - rich communication primitives, no primitive features
    for fault-tolerance
  \item ML5 - code mobility, type-safe marshalling, etc, but no
    primitive features for fault-tolerance
  \item Glasgow Distributed Haskell - features for fault-tolerance,
    error detection/recovery
  \end{itemize}


  Libraries for existing functional languages --- like Functory
  \begin{itemize}
  \item Plasma MR - OCaml implementation of Map/Reduce
  \item iTask - Clean library for distributed workflow management
  \end{itemize}

% Our library could be implemented using any of the DFLs, but our goals
% are:
% \begin{itemize}
% \item Provide users of an existing language, user-friendly APIs
% \item Provide robust fault-tolerance mechanism 
% \item Relieve the users from coding messy details
% \end{itemize}
  
\end{frame}

%% -------------------------------------------------------------------- %%

\begin{frame}\frametitle {Future Work}
  \begin{itemize}
  \item More user control
    \begin{itemize}
    \item Scheduling of tasks, etc
    \end{itemize}


  \item Real-time visualization 
    \begin{itemize}
    \item Resource consumption, task distribution patterns, etc.
    \end{itemize}


  \item Speeding up using idle workers
  \end{itemize}
\end{frame}

%% -------------------------------------------------------------------- %%

\begin{frame}\frametitle {Thanks}


Check out:
\begin{center}
 \textcolor{blue}{\url{http://functory.lri.fr/}}  
\end{center}

Feedback, comments welcome!

\end{frame}

%% -------------------------------------------------------------------- %%
\end{document}



%%% Local Variables: 
%%% mode: latex
%%% TeX-master: t
%%% End: 

